%------------------ настройки шрифтов и геометрии листа -------------------------------
\documentclass[12pt]{article} % шрифт и формат документа
\usepackage[a4paper, left=30mm, right=10mm, top=20mm, bottom=20mm]{geometry} % отступы от края листа
\usepackage{fontspec} % шрифт таймс
\usepackage{polyglossia}
\usepackage{amsmath}
\usepackage{amssymb}
\usepackage{graphicx}
\usepackage{lipsum} % для генерации dummy текста
\usepackage{pdfpages} %  для вставки ПДФ
\usepackage{longtable}
\usepackage{enumitem}
\usepackage[labelsep=endash, font={normalsize, stretch=1}, justification=centering]{caption}
\usepackage[titles]{tocloft}
\usepackage{titlesec} % для контроля над заголовками
\bibliographystyle{gost-numeric.bbx}
\usepackage[parentracker=true,
backend=biber,
hyperref=false,
bibencoding=utf8,
style=numeric-comp,
language=auto,
autolang=other,
citestyle=gost-numeric,
defernumbers=true,
bibstyle=gost-numeric,
%sorting=none, %эти два пакета почему-то ломают библиографию
]{biblatex}

\setmainlanguage{russian}
\setmainfont[]{Times New Roman}
\renewcommand{\baselinestretch}{1.5} % межстрочный интервал 1.5 (1.6 потому что шрифты )
\setlength\parindent{1.25cm} % абзацный отступ

%------------------ настройка списков и подписей ----------------------------------------

\setlist[enumerate]{label*=\arabic*., leftmargin=(\parindent+16pt), itemsep={0cm}, parsep={0cm}} % пакет для нормальных списков с подпунктами и абзацным отступом
\setlength\parsep{0cm}
\setlength\itemsep{-1cm}
% почему-то настройка слагаемых leftmargin не работает, поэтому подход кривой 

%\addto\captionsrussian{\renewcommand{\figurename}{Рисунок}}
\captionsrussian{\renewcommand{\figurename}{Рисунок}}

\captionsetup[table]{justification=RaggedRight, position=top, singlelinecheck=off}
%------------------ настройки ToC -------------------------------------------------------
\renewcommand{\cftsecleader}{\cftdotfill{\cftdotsep}} % точки в содержании
\AtBeginDocument{\renewcommand\contentsname{}}
\addto\captionsrussian{\renewcommand{\contentsname}{\hfill Содержание \hfill}} % Меняю подпись к содержанию (так лучше чем с секцией)

\renewcommand{\cftsecfont}{} % шрифт содержания
\renewcommand{\cftsecpagefont}{}
%\renewcommand{\cftsubsecfont}{\normalfont}

\setlength\cftbeforesecskip{0pt}

%------------------- настройки заголовков----------------------------------------------

\newcommand{\sectionbreak}{\clearpage} % перенос на новую страницу \section
\let\oldsection\section 
\renewcommand\section{\clearpage\oldsection} % перенос на новую страницу структурных элементов \section*

\titleformat{\subsection} % меняю начертание подзаголовка
{\normalfont\bfseries\Large}{\thesubsection}{1em}{}
\titlespacing*{\subsection}{\parindent}{1ex}{1em}
\titleformat{\section} % меняю начертание подзаголовка
{\normalfont\bfseries\Large}{\thesection}{1em}{}
\titlespacing*{\section}{\parindent}{1ex}{1em}

%------------------- кастомные секции для стркутрных элементов -----------------------
\makeatletter
\newcommand\@csection{\@startsection {section}{1}{\z@}%
	{-3.5ex \@plus -1ex \@minus -.2ex}%
	{2.3ex \@plus.2ex}%
	{\normalfont\Large\bfseries\centering}}
\newcommand{\csection}[1]{%
	\clearpage
	\@csection*{#1}%
	\addcontentsline{toc}{section}{#1}%
}
\makeatother

\makeatletter
\newcommand\@csubsection{\@startsection {section}{1}{\z@}%
	{-3.5ex \@plus -1ex \@minus -.2ex}%
	{2.3ex \@plus.2ex}%
	{\normalfont\Large\bfseries\centering}}
\newcommand{\csubsection}[1]{%
	\@csubsection*{#1}%
	\addcontentsline{toc}{subsection}{#1}%
}
\makeatother

%------------------- библиография -----------------------------------------------------
\addbibresource{\jobname.bib}

\newcommand{\fullbibtitle}{}