\section{Основная часть раз}
\lipsum[1] \cite{Labov1972}
\subsection{Подчасть раз}
Перечисляю плюсы использования \LaTeX: 
\begin{enumerate}
	\item Это изматывает,
	\item Это плохо сказывается на моей психике, 
	\item Так делали великие люди: 
	\begin{enumerate}
		\item Заратустра,
		\item Кант,
		\item Леонид Андрейченко.
	\end{enumerate}
	\item \lipsum[2][1-3]
\end{enumerate}

\section{Основная часть два}
\lipsum[2-3] 

\begin{longtable}{|p{0.48\linewidth}|p{0.48\linewidth}|} %ширина таблицы должна быть 0.96\linewidth (почему-то не 1)
	\caption{Очень длинная таблица}
	\label{tab:table1}
	\endfirsthead
	\caption*{Продолжение таблицы \ref{tab:table1}}
	\endhead
	\hline
	\hfil Заголовок 1 \hfil & \hfil Заголовок 2 \hfil\\ \hline
	\lipsum[4][1]  & \lipsum[4][2] \\ \hline
	\lipsum[4][3]  & \lipsum[4][4] \\ \hline
	\lipsum[4][5]  & \lipsum[4][6] \\ \hline
	\lipsum[4][1]  & \lipsum[4][2] \\ \hline
	\lipsum[4][3]  & \lipsum[4][4] \\ \hline
	\lipsum[4][5]  & \lipsum[4][6] \\ \hline
	\lipsum[4][1]  & \lipsum[4][2] \\ \hline
	\lipsum[4][3]  & \lipsum[4][4] \\ \hline
	\lipsum[4][5]  & \lipsum[4][6] \\ \hline
	\lipsum[4][1]  & \lipsum[4][2] \\ \hline
	\lipsum[4][3]  & \lipsum[4][4] \\ \hline
	\lipsum[4][5]  & \lipsum[4][6] \\ \hline
	\lipsum[4][1]  & \lipsum[4][2] \\ \hline
	\lipsum[4][3]  & \lipsum[4][4] \\ \hline
	\lipsum[4][5]  & \lipsum[4][6] \\ \hline
	\lipsum[4][1]  & \lipsum[4][2] \\ \hline
	\lipsum[4][3]  & \lipsum[4][4] \\ \hline
	\lipsum[4][5]  & \lipsum[4][6] \\ \hline
	\lipsum[4][1]  & \lipsum[4][2] \\ \hline
	\lipsum[4][3]  & \lipsum[4][4] \\ \hline
	\lipsum[4][5]  & \lipsum[4][6] \\ \hline
\end{longtable}


Теперь я сошлюсь на рисунок \ref{fig:56376d9aa8d6b}. Если зажать ctrl, можно даже перейти по ссылке \cite{Chomsky1957}
\begin{figure}[hb]
	\centering
	\includegraphics[width=0.8\linewidth]{inc/img/56376d9aa8d6b}
	\caption{Смешной котик. Надеюсь, до этого не дойдет, но нужно проверить междустрочный интервал}
	\label{fig:56376d9aa8d6b}
\end{figure}